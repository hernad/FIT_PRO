\documentclass[times, utf8, seminar]{fit}

%\batchmode
%\usepackage{booktabs}
\usepackage{listings}
\usepackage{longtable}
\usepackage{xcolor}
\usepackage{float}
\usepackage{enumitem}
\usepackage{hyperref}
\usepackage{enumerate}

\begin{document}

\title{Projekat: Unapređenje informacionog sistema na primjeru jedne kompanije}

\author{Ernad Husremović}
\brindex{DL 2792}
\verzija {1.0.0}

\mentor{prof.dr Murat Prašo}

\maketitle

\tableofcontents

% abstract begin
\begin{sazetak}


Sažetak na bosanskom jeziku.

\kljucnerijeci{bosanska rijec 1, bosanska rijec 2}
\end{sazetak}

\engtitle{Naslov na engleskom jeziku}
\begin{abstract}
Abstract na engleskom jeziku.

\keywords{key1, key2, key3}
\end{abstract}

% abstract end



\chapter{Uvod}

\section{Legalizacija - bezpovratne promjene na IT tržištu BiH}

Akcije državnih inspekcijskih organa na suzbijanju nelegalnog softvera (nadalje legalizacija) sa početkom 2012 godine označila je početak krupnih promjena na tržištu informatičkih \engl{information technology, nadalje IT} rješenja u Bosni i Hercegovini. 
Većina poslovnih subjekata koja se u proteklim godinama opremila "jeftinim" nelegalnim softverom \engl{software} našla se u nezavidnoj situaciji.
Izloženi visokim troškovima legalizacije svuda zastupljenog "Microsoft Windows" software-a, pojavila se potražnja za legalnim softverom drugih proizvođača.  

\section{OSS informatička rješenja}

Preduzeće "bring.out"\footnote{\url{http://www.bring.out.ba/}} je dugi niz godina orjentisano na ponudu sistema baziranih na software-u otvorenog koda \engl{open source software, nadalje OSS}.

Linux/Ubuntu serverska rješenja \footnote{\url{http://linux.com}, \url{http://ubuntu.com}}) su i prije legalizacije imali svoje tržište. Međutim, nakon legalizacije pojavila se potražnja za cjelovitim informatičkim IT rješenjima baziranim na OSS-u:
\begin{itemize}
  \item server operativni sistem (nadalje OS)
  \item desktop operativni sistem
  \item aplikativni softver
  \begin{itemize}
    \item opći software za uredske potrebe
    \item poslovni - softver \engl{Enterprise Resource Planning, nadalje ERP}
  \end{itemize}
\end{itemize}

Ovaj projekat će prezentovati instalaciju i troškove eksploatacije jednog takvog sistema.

Komparativnom analizom utvrdićemo toškove sličnog sistema baziranog na zatvorenim tehnologijama:
\begin{enumerate}[(a)]
  \item sa uključenim troškovima legalizacije postojećeg nelegalnog "Windows" OS softvera bez zamjene postojećeg ERP aplikativnog softvera
  \item sa kompletnom zamjenom kako sistemskog, tako i aplikativnog ERP softvera.  
\end{enumerate}

Korisnik je mala kompanija koja se bavi trgovinom. Sa stanovišta IT radnih mjesta kompanija se sastoji od:
\begin{itemize}
  \item maloprodajna mreža od 5 prodavnica
  \item centrala kompanije sa knjigovodstvom, 4 radne stanice
  \item 1 server baze podataka 
\end{itemize}


\section{Test}

Test

Nabrajalica

\begin{enumerate}
  \item item 1
  \item item 2
\end{enumerate}

bla bla \footnote{fus nota}.

\subsection{Podnaslov neki}

\label{labela_oznaka}

\begin{description}
  \item [item 1]  objasni item 1
  \item [item 2] objasni item 2
\end{description}
 

Citacija \cite[str.~391]{pentaho32}.

\begin{itemize}
   \item vako 
   \item nako
\end{itemize}

\begin{enumerate}
  \item \emph{Extract:}  uzimanje podataka iz vanjskih izvora
  \item \emph{Transform:} transformacija operativnih podataka u format koji je pogodan za pohranu u skladište podataka (vidi \ref{labela_oznaka})
  \item \emph{Load:} snimanje `prečišćenih' podataka u skladište podataka (DW/DMart)
\end{enumerate}
 

Referenca na sekciju sekcija1 (vidi \ref{sect:sekcija1}).

 
\begin{figure}[H]
\centering
\includegraphics[width=12cm]{img/pentaho_arhitektura_eric.png}
\caption{bla bla (\cite{web:eric})}
\end{figure}

\section{Sekcija 1 sa slikom}
\label{sect:sekcija1}

\begin{figure}[H]
\centering
\includegraphics[width=15cm]{img/vako_nako.pdf}
\caption{vako nako caption}
\end{figure}


\section{Zaključak}

\bibliography{literatura}
\bibliographystyle{fit}

\appendix

\chapter{dodatak 1}
\label{chap:dodatak1}

\chapter{Bilješke}
\label{chap:biljeske}

\begin{enumerate}
  \item jedan
  \item dva: \url{http://redmine.bring.out.ba/issues/26711}
\end{enumerate}

\end{document}
